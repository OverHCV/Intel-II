\chapter{Trabajo Relacionado}
\label{chap:related-work}

% TODO: Agregar citas bibliográficas específicas para cada sección

Este capítulo presenta una revisión del estado del arte en detección y clasificación de malware, con énfasis en enfoques basados en aprendizaje automático y Deep Learning. Se analizan los métodos tradicionales, las técnicas emergentes basadas en visión por computador, y los trabajos previos que fundamentan la propuesta de este proyecto.

% =============================================================================
\section{Métodos Tradicionales de Detección de Malware}
\label{sec:related:traditional}

\subsection{Detección Basada en Firmas}

% TODO: Citar antivirus comerciales y estudios académicos
Los sistemas antivirus tradicionales emplean detección basada en firmas, donde se comparan patrones binarios conocidos (hashes o secuencias de bytes) contra bases de datos de malware identificado. Este enfoque presenta alta precisión para amenazas conocidas pero falla ante:
\begin{itemize}
    \item \textbf{Polimorfismo:} Malware que modifica su código en cada infección
    \item \textbf{Ofuscación:} Técnicas de empaquetado que alteran la firma sin cambiar funcionalidad
    \item \textbf{Zero-day:} Amenazas completamente nuevas sin firma registrada
\end{itemize}

\subsection{Análisis Heurístico}

Los métodos heurísticos evalúan comportamientos sospechosos mediante reglas predefinidas, como acceso no autorizado al registro, modificación de archivos del sistema, o comunicaciones de red anómalas. Aunque más flexibles que las firmas, dependen de la experticia humana para definir reglas y pueden generar altas tasas de falsos positivos.

\subsection{Análisis Dinámico (Sandboxing)}

% TODO: Referenciar herramientas como Cuckoo Sandbox
El análisis dinámico ejecuta muestras sospechosas en entornos controlados (sandboxes) para observar su comportamiento. Proporciona información detallada pero presenta desventajas:
\begin{itemize}
    \item Alto costo computacional y temporal
    \item Malware puede detectar entornos virtualizados y alterar su comportamiento
    \item Dificultad para escalar a análisis masivo
\end{itemize}

% =============================================================================
\section{Machine Learning en Detección de Malware}
\label{sec:related:machine-learning}

\subsection{Enfoques Tradicionales de ML}

% TODO: Citar trabajos seminales de ML para malware
La aplicación de técnicas de Machine Learning a la clasificación de malware ha sido ampliamente estudiada. Los enfoques clásicos incluyen:

\textbf{Support Vector Machines (SVM):} Utilizadas para clasificación binaria (benigno vs. malicioso) y multi-clase (familias de malware) basándose en características extraídas manualmente como n-gramas de bytes, llamadas API, o estructuras PE.

\textbf{Random Forests y Decision Trees:} Algoritmos ensemble que combinan múltiples árboles de decisión, efectivos para capturar relaciones no lineales entre características.

\textbf{k-Nearest Neighbors (k-NN):} Clasificación basada en similitud con muestras conocidas, útil para detección de variantes de familias existentes.

La principal limitación de estos métodos radica en la dependencia de ingeniería manual de características (feature engineering), proceso que requiere expertise de dominio y puede no capturar representaciones óptimas.

% =============================================================================
\section{Deep Learning para Análisis de Malware}
\label{sec:related:deep-learning}

\subsection{Ventajas del Deep Learning}

% TODO: Añadir referencias a surveys de DL en ciberseguridad
El Deep Learning ofrece capacidad de aprendizaje automático de representaciones jerárquicas directamente de datos crudos o mínimamente procesados. Las principales arquitecturas aplicadas incluyen:

\textbf{Redes Neuronales Convolucionales (CNN):} Especialmente efectivas cuando el malware se representa como imágenes o secuencias con estructura espacial.

\textbf{Redes Neuronales Recurrentes (RNN/LSTM):} Utilizadas para analizar secuencias de instrucciones o llamadas API, capturando dependencias temporales.

\textbf{Autoencoders:} Empleados para detección de anomalías, identificando muestras que se desvían significativamente de patrones normales.

\subsection{Trabajos Previos con Deep Learning}

% TODO: Revisar y citar papers específicos de cada enfoque
Diversos estudios han explorado Deep Learning para malware:

\begin{itemize}
    \item Análisis de secuencias de bytes mediante CNN 1D
    \item Clasificación de malware de Android mediante análisis de permisos y grafos de llamadas
    \item Detección de ransomware usando autoencoders variacionales
    \item Clasificación de familias mediante embeddings aprendidos
\end{itemize}

% =============================================================================
\section{Análisis Visual de Malware}
\label{sec:related:visual-analysis}

\subsection{Conversión de Binarios a Imágenes}

% TODO: Citar trabajos seminales de visualización de malware (Nataraj et al.)
Un enfoque innovador consiste en visualizar ejecutables como imágenes. El proceso típico incluye:

\begin{enumerate}
    \item Leer el archivo binario como secuencia de bytes
    \item Interpretar cada byte como valor de intensidad de píxel (0-255)
    \item Organizar los píxeles en una matriz bidimensional
    \item Aplicar redimensionamiento a tamaño estándar
\end{enumerate}

Esta representación visual preserva la estructura del ejecutable y permite aplicar técnicas de visión por computador.

\subsection{Características Visuales Discriminativas}

% TODO: Explicar qué patrones visuales distinguen familias de malware
Estudios han demostrado que diferentes familias de malware exhiben texturas visuales distintivas relacionadas con:
\begin{itemize}
    \item Estructura del código (secciones .text, .data, .rdata)
    \item Técnicas de empaquetado y compresión
    \item Presencia de recursos embebidos (imágenes, configuraciones)
    \item Patrones de cifrado o ofuscación
\end{itemize}

Las CNN son particularmente efectivas para extraer automáticamente estas características visuales.

% =============================================================================
\section{Trabajos con CNN para Clasificación de Malware}
\label{sec:related:cnn-malware}

\subsection{Arquitecturas CNN Aplicadas}

% TODO: Detallar arquitecturas específicas usadas en literatura
Diversos trabajos han aplicado CNN para clasificación de malware visual:

\textbf{CNN Shallow:} Arquitecturas simples de pocas capas para datasets pequeños, reduciendo riesgo de overfitting.

\textbf{CNN Profundas Personalizadas:} Arquitecturas diseñadas específicamente para características de imágenes de malware.

\textbf{Transfer Learning:} Uso de modelos preentrenados (VGG, ResNet, Inception) en ImageNet, adaptados mediante fine-tuning para malware.

\subsection{Resultados Reportados en Literatura}

% TODO: Incluir tabla comparativa con trabajos previos y sus resultados
Los estudios existentes reportan precisiones que varían entre 85\% y 99\% dependiendo de:
\begin{itemize}
    \item Complejidad del dataset (número de familias)
    \item Calidad y cantidad de muestras de entrenamiento
    \item Arquitectura CNN utilizada
    \item Técnicas de regularización y aumento de datos
\end{itemize}

% =============================================================================
\section{Datasets Públicos de Malware}
\label{sec:related:datasets}

\subsection{Descripción de Datasets Relevantes}

% TODO: Detallar características específicas de cada dataset usado
Los principales datasets públicos para investigación en clasificación de malware incluyen:

\textbf{MalImg Dataset:} Contiene imágenes de malware organizadas en 25 familias, ampliamente utilizado como benchmark estándar.

\textbf{Malevis Dataset:} Dataset más reciente con mayor variedad de familias y muestras, incluyendo malware moderno.

\textbf{Blended Malware Image Dataset:} Combinación de múltiples fuentes, ofreciendo diversidad en tipos y familias.

Estos datasets proporcionan la base empírica para entrenar y evaluar modelos de clasificación.

% =============================================================================
\section{Brechas en la Literatura y Motivación del Proyecto}
\label{sec:related:gaps}

% TODO: Refinar según contribuciones específicas del proyecto
A pesar de los avances, persisten oportunidades de investigación:

\begin{itemize}
    \item \textbf{Comparación sistemática:} Pocos estudios comparan múltiples datasets y arquitecturas bajo condiciones controladas.

    \item \textbf{Interpretabilidad:} Limitada comprensión de qué características visuales específicas utiliza la CNN para discriminar familias.

    \item \textbf{Robustez:} Evaluación insuficiente ante técnicas adversariales diseñadas para engañar clasificadores.

    \item \textbf{Escalabilidad:} Necesidad de validar rendimiento en datasets masivos más representativos de entornos reales.
\end{itemize}

Este proyecto busca contribuir mediante una evaluación exhaustiva de múltiples arquitecturas CNN sobre tres datasets reconocidos, con análisis detallado de rendimiento y características aprendidas.

% =============================================================================
\section{Resumen}
\label{sec:related:summary}

La revisión del estado del arte revela que:
\begin{enumerate}
    \item Los métodos tradicionales de detección presentan limitaciones significativas ante malware moderno.
    \item El Machine Learning y Deep Learning han demostrado gran potencial para clasificación automatizada.
    \item El enfoque visual de malware permite aplicar técnicas probadas de visión por computador.
    \item Existen datasets públicos adecuados para entrenar y evaluar modelos.
    \item Persisten oportunidades para investigación adicional en comparación de arquitecturas e interpretabilidad.
\end{enumerate}

Estos fundamentos justifican el enfoque propuesto en este proyecto, combinando análisis visual con CNN para clasificación robusta y eficiente de familias de malware.
